\documentclass[a4paper, 11pt]{article}
\usepackage[margin=2cm]{geometry}
\linespread{1.5}
\setlength{\parindent}{0mm}
\setlength{\parskip}{2mm}

\pagenumbering{gobble}

\begin{document}

\hfill Balduin Dettling

\vspace{1cm}
\begin{center}

	{\huge\bfseries Development of a 3D Display\vspace{-2mm}}
	\rule{0.67\textwidth}{1.5pt}

	{\vspace{-2mm} \Large\bfseries Straightforward Description}

\end{center}

In this project, I have built a three-dimensional persistence of vision
display. Unlike most 3D displays (like television or cinema screens), it's a
\emph{volumetric} display, which means that it actually displays depth instead
of merely providing an illusion of it.

It works by quickly rotating 160 RGB LEDs. A microcontroller controls them such
that they draw an image into the air. This happens so fast that the human eye
doesn't see quick bursts of light, but steadily lit pixels.

\end{document}
