\documentclass[a4paper, 11pt]{article}
\usepackage[margin=2cm]{geometry}
\linespread{1.5}
\setlength{\parindent}{0mm}
\setlength{\parskip}{2mm}

\usepackage{siunitx}

\pagenumbering{gobble}

\begin{document}

\hfill Balduin Dettling

\vspace{1cm}
\begin{center}

    {\huge\bfseries Development of a 3D Display\vspace{-2mm}} 
    \rule{0.67\textwidth}{1.5pt}

    {\vspace{-2mm} \Large\bfseries Summary}

\end{center}

\section*{Goals}

The aim of this project was to build a working prototype of a volumetric persistence of vision
display.

\section*{Methods}

The display consists of a two-dimensional arrangement of LEDs that rotates at about
\SI{60}{\hertz}. A microcontroller controls the LEDs such that they produce a volumetric image
that can be viewed from any direction. Using a hall effect sensor and a magnet, the display
measures its rotational speed and adjusts its timing accordingly, so that the image stays in place
despite the rotational frequency slightly varying. 

Because the rotational period is smaller than the flicker fusion threshold of the human eye, the
image appears to be steady and the observer sees the whole image at all times, even though at any
given time only $\frac{1}{100}$ of the pixels are illuminated.

A python program makes it easy to insert simple geometric shapes or plots of a user-given function
into an image, and it can convert the generated images to programs for the microcontroller that
display the image.

\section*{Conclusions}

The initial goal has been reached and the display works as planned. Possible applications include
advertising, education and video games. My display could be improved by incorporating wireless
data transmission, so that images can be changed at runtime, or by replacing the loud and
inefficient slip rings with inductive power transfer. Another worthwile addition would be a
program that converts 3D models in a popular format to the format used by my display, which is
nothing but a 3D bitmap.

\end{document}
